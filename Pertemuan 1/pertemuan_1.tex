\documentclass{article}
\usepackage{fancyhdr}
\usepackage{titlesec}
\usepackage{graphicx}
\graphicspath{ {./img/} }
\usepackage{multirow}
\usepackage{listings}
\usepackage{xcolor}
% \usepackage[a4paper, portrait, margin=1in]{geometry}

\definecolor{codegreen}{rgb}{0,0.6,0}
\definecolor{codegray}{rgb}{0.5,0.5,0.5}
\definecolor{codepurple}{rgb}{0.58,0,0.82}
\definecolor{backcolour}{rgb}{0.95,0.95,0.92}

\lstdefinestyle{pythonstyle}{
    backgroundcolor=\color{backcolour},   
    commentstyle=\color{codegreen},
    keywordstyle=\color{magenta},
    numberstyle=\tiny\color{codegray},
    stringstyle=\color{codepurple},
    basicstyle=\ttfamily\footnotesize,
    breakatwhitespace=false,         
    breaklines=true,                 
    captionpos=b,                    
    keepspaces=true,                 
    numbers=left,                    
    numbersep=5pt,                  
    showspaces=false,                
    showstringspaces=false,
    showtabs=false,                  
    tabsize=2
}

\lstdefinestyle{bashstyle}{
    backgroundcolor=\color{backcolour},   
    commentstyle=\color{codegreen},
    keywordstyle=\color{magenta},
    numberstyle=\tiny\color{codegray},
    stringstyle=\color{codepurple},
    basicstyle=\ttfamily\footnotesize,
    breakatwhitespace=false,         
    breaklines=true,                 
    captionpos=b,                    
    showspaces=false,                
    showstringspaces=false,
    showtabs=false,                  
    tabsize=2
}

\pagestyle{fancy}
\fancyhf{}
\lhead{Modul 1 Praktikum Natural Language Processing}
\rfoot{\footnotesize Page \thepage}
\lfoot{\footnotesize Prof. Dr. Taufik Fuadi Abidin, S.Si., M. Tech. \newline Jurusan Informatika Universitas Syiah Kuala \newline Modul oleh : Diky Wahyudi, Furqan Al Ghifary Zulfa}
\renewcommand{\headrulewidth}{1pt}
\renewcommand{\footrulewidth}{1pt}

\titleformat*{\section}{\small\bfseries}

\begin{document}
    \begin{center}
        \textbf{Modul 1 Praktikum Natural Language Processing}

        \textbf{Pengenalan Pengolahan Data Teks menggunakan Python}
    \end{center}

    \section*{Deskripsi Singkat}
    Deskripsi

    \section*{Tujuan}
    \begin{enumerate}
        \item Tujuan
        \item Tujuan
        \item Tujuan
    \end{enumerate}

    \begin{flushleft}
        \textbf{Materi 1 - Pengumpulan Data}
        \newline
        
        \textbf{Studi Kasus}
        \newline
        Anda ingin mengumpulkan data menggunakan API Twitter
        \newline

        \textbf{Langkah-langkah}

        \begin{enumerate}
            \item Menginstall library tweepy
            \lstset{style=bashstyle}
            \begin{lstlisting}[language=bash]
pip install tweepy
            \end{lstlisting}

            \item Membuat script python untuk mengumpulkan data
            \lstset{style=pythonstyle}
            \begin{lstlisting}[language=python]
import tweepy
import numpy as np

            \end{lstlisting}
        \end{enumerate}
    \end{flushleft}

    \begin{flushleft}
        \textbf{Materi 2 - Membaca data dari PDF}
        \newline

        Isi materi 2
    \end{flushleft}

    \begin{flushleft}
        \textbf{Materi 3 - Membaca data dari File Word (Microsoft Word)}
        \newline

        Isi materi 3
    \end{flushleft}

    \begin{flushleft}
        \textbf{Materi 4 - Membaca data dari JSON}
        \newline

        Isi materi 4
    \end{flushleft}

    \begin{flushleft}
        \textbf{Materi 5 - Membaca data dari HTML}
        \newline

        Isi materi 3
    \end{flushleft}

    \newpage
    \begin{flushleft}
        \textbf{Tugas}
        \newline

        \begin{enumerate}
            \item Kerjakan semua materi 1 - 5 diatas
        \end{enumerate}
    \end{flushleft}
\end{document}