\documentclass{article}
\usepackage{fancyhdr}
\usepackage{titlesec}
\usepackage{graphicx}
\graphicspath{ {./img/} }
\usepackage{multirow}
\usepackage{listings}
\usepackage{xcolor}
\usepackage{hyperref}
\usepackage{blindtext}
\hypersetup{
    colorlinks=true,
    linkcolor=blue,
    filecolor=magenta,
    urlcolor=blue,
}
% \usepackage[a4paper, portrait, margin=1in]{geometry}

\definecolor{codegreen}{rgb}{0,0.6,0}
\definecolor{codegray}{rgb}{0.5,0.5,0.5}
\definecolor{codepurple}{rgb}{0.58,0,0.82}
\definecolor{backcolour}{rgb}{0.95,0.95,0.92}

\lstdefinestyle{pythonstyle}{
    backgroundcolor=\color{backcolour},   
    commentstyle=\color{codegreen},
    keywordstyle=\color{magenta},
    numberstyle=\tiny\color{codegray},
    stringstyle=\color{codepurple},
    basicstyle=\ttfamily\footnotesize,
    breakatwhitespace=false,         
    breaklines=true,                 
    captionpos=b,                    
    numbers=left,                    
    numbersep=5pt,                  
    showspaces=false,                
    showstringspaces=false,
    showtabs=false,                  
    tabsize=2
}

\lstdefinestyle{bashstyle}{
    backgroundcolor=\color{backcolour},   
    commentstyle=\color{codegreen},
    keywordstyle=\color{magenta},
    numberstyle=\tiny\color{codegray},
    stringstyle=\color{codepurple},
    basicstyle=\ttfamily\footnotesize,
    breakatwhitespace=false,         
    breaklines=true,                 
    captionpos=b,                    
    showspaces=false,                
    showstringspaces=false,
    showtabs=false,                  
    tabsize=2
}

\pagestyle{fancy}
\fancyhf{}
\lhead{Modul 1 Praktikum Natural Language Processing}
\rfoot{\footnotesize Page \thepage}
\lfoot{\footnotesize Prof.\@ Dr.\@ Taufik Fuadi Abidin, S.Si., M. Tech\@. \newline Jurusan Informatika Universitas Syiah Kuala \newline Modul oleh: Diky Wahyudi, Furqan Al Ghifari Zulva}
\renewcommand{\headrulewidth}{1pt}
\renewcommand{\footrulewidth}{1pt}

\titleformat*{\section}{\small\bfseries}

\begin{document}
    \begin{center}
        \textbf{Modul 1 Praktikum Natural Language Processing}

        \textbf{Pengenalan Pengolahan Data Teks menggunakan Python}
    \end{center}

    \section*{Deskripsi Singkat}
    Modul ini akan membahas pengenalan pengolahan data teks menggunakan bahasa Python. Modul ini terdiri dari 5 materi yang akan membahas pengumpulan data menggunakan API, membaca data dari berbagai format file yang ada seperti PDF, Word, JSON, dan HTML\@.

    \section*{Tujuan}
    \begin{enumerate}
        \item Dapat mengumpulkan data menggunakan melalui API yang tersedia di internet.
        \item Dapat membaca data dari berbagai format file yang ada seperti PDF, Word, JSON, dan HTML.\@
    \end{enumerate}

    \begin{flushleft}
        \textbf{Materi 1 \@- Pengumpulan Data}
        \newline

        \textbf{Studi Kasus}
        \newline
        Anda ingin mengumpulkan data menggunakan API Twitter
        \newline

        \textbf{Langkah-langkah}

        \begin{enumerate}
            \item Mendapatkan Bearer Token pada halaman Portal Developer Twitter \href{https://developer.x.com/en/portal/}{https://developer.x.com/en/portal/}
            \item Menginstall library tweepy
            \lstset{style=bashstyle}
            \begin{lstlisting}[language=bash]
pip install tweepy
            \end{lstlisting}

            \item Import library yang dibutuhkan
            \lstset{style=pythonstyle}
            \begin{lstlisting}[language=python]
import tweepy
            \end{lstlisting}

            \item Memasukkan bearer token sebagai akses token ke API Twitter
            \lstset{style=pythonstyle}
            \begin{lstlisting}[language=python]
consumer_key = "your_consumer_key"
consumer_secret = "your_consumer_secret"
access_token = "your_access_token"
access_token_secret = "your_access_token_secret"
bearer_token = 'your_bearer_token'
            \end{lstlisting}

            \item Membuat client ke API Twitter
            \lstset{style=pythonstyle}
            \begin{lstlisting}[language=python]
client = tweepy.Client(bearer_token=bearer_token, consumer_key=consumer_key, consumer_secret=consumer_secret, access_token=access_token, access_token_secret=access_token_secret)            \end{lstlisting}

            \item Membuat query untuk mengambil data tweet berdasarkan kata kunci
            \lstset{style=pythonstyle}
            \begin{lstlisting}[language=python]
query = 'covid-19'
result = client.search_recent_tweets(query=query, max_results=20)
            \end{lstlisting}
        \end{enumerate}
    \end{flushleft}

    \begin{flushleft}
        \textbf{Materi 2 \@- Membaca data dari PDF}
        \newline

        \textbf{Studi Kasus}
        \newline
        Anda ingin membaca data dari file PDF
        \newline

        \textbf{Langkah-langkah}

        \begin{enumerate}
            \item Menginstall library PyPDF2
            \lstset{style=bashstyle}
            \begin{lstlisting}[language=bash]
pip install PyPDF2
            \end{lstlisting}

            \item Import library yang dibutuhkan
            \lstset{style=pythonstyle}
            \begin{lstlisting}[language=python]
import PyPDF2
from PyPDF2 import PdfReader
            \end{lstlisting}

            \item Membaca file PDF
            \lstset{style=pythonstyle}
            \begin{lstlisting}[language=python]
# Membuka file PDF
pdf = open("pertemuan_1.pdf", "rb")

# Membuat objek pembaca PDF
pdf_reader = PyPDF2.PdfReader(pdf)

# Mengecek jumlah halaman dalam file PDF
print(pdf_reader.pages)

# Mendapatkan object sebuah halaman
page = pdf_reader.pages[0]

# Mengekstrak teks dari halaman
print(page.extract_text())

# Menutup file PDF setelah selesai
pdf.close()
            \end{lstlisting}
        \end{enumerate}
    \end{flushleft}

    \begin{flushleft}
        \textbf{Materi 3 \@- Membaca data dari File Word (Microsoft Word)}
        \newline

        \textbf{Studi Kasus}
        \newline
        Anda ingin membaca data dari file Word
        \newline

        \textbf{Langkah-langkah}

        \begin{enumerate}
            \item Menginstall library python-docx
            \lstset{style=bashstyle}
            \begin{lstlisting}[language=bash]
pip install docx # Library untuk python versi lama
pip install python-docx # Library terbaru
            \end{lstlisting}

            \item Import library yang dibutuhkan
            \lstset{style=pythonstyle}
            \begin{lstlisting}[language=python]
import docx
            \end{lstlisting}

            \item Membaca text dari file Word
            \lstset{style=pythonstyle}
            \begin{lstlisting}[language=python]
# Membaca text dari file Word
document = docx.Document("file.docx")

text = ""

for para in document.paragraphs:
    text += para.text

# Cetak hasil text dari file Word
print(text)
            \end{lstlisting}
        \end{enumerate}
    \end{flushleft}

    \begin{flushleft}
        \textbf{Materi 4 \@- Membaca data dari JSON}
        \newline

        \textbf{Studi Kasus}
        \newline
        Anda ingin membaca data dari file JSON dengan fetch data dari API
        \newline

        \textbf{Langkah-langkah}

        \begin{enumerate}
            \item Import library yang dibutuhkan
            \lstset{style=pythonstyle}
            \begin{lstlisting}[language=python]
import requests
import json
            \end{lstlisting}

            \item Melakukan fetch data dari API
            \lstset{style=pythonstyle}
            \begin{lstlisting}[language=python]
# Fetch data dari API
r = requests.get("https://jsonplaceholder.typicode.com/todos")
res = r.json()

# Cetak hasil fetch data dari API
print(res)
            \end{lstlisting}

            \item Melakukan ekstraksi data dari JSON
            \lstset{style=pythonstyle}
            \begin{lstlisting}[language=python]
#extract contents
content = res[0]['title']
print(content)
            \end{lstlisting}
        \end{enumerate}

        atau jika data yang ingin diambil berada di dalam file JSON

        \begin{enumerate}
            \item Membaca data dari file JSON
            \lstset{style=pythonstyle}
            \begin{lstlisting}[language=python]
# Membaca data dari file JSON
with open('data.json') as f:
    data = json.load(f)
            \end{lstlisting}

            \item Melakukan ekstraksi data dari JSON
            \lstset{style=pythonstyle}
            \begin{lstlisting}[language=python]
#extract contents
content = data[0]['title']
print(content)
            \end{lstlisting}
        \end{enumerate}
    \end{flushleft}

    \begin{flushleft}
        \textbf{Materi 5 \@- Membaca data dari HTML}
        \newline

        \textbf{Studi Kasus}
        \newline
        Anda ingin membaca data dari file HTML
        \newline

        \textbf{Langkah-langkah}

        \begin{enumerate}
            \item Menginstall library BeautifulSoup
            \lstset{style=bashstyle}
            \begin{lstlisting}[language=bash]
pip install beautifulsoup4
            \end{lstlisting}

            \item Import library yang dibutuhkan
            \lstset{style=pythonstyle}
            \begin{lstlisting}[language=python]
from bs4 import BeautifulSoup   
import urllib.request as urllib2
            \end{lstlisting}

            \item Mengambil file HTML melalui URL
            \lstset{style=pythonstyle}
            \begin{lstlisting}[language=python]
response = urllib2.urlopen('https://en.wikipedia.org/wiki/Natural_language_processing')
html_doc = response.read()
            \end{lstlisting}

            \item Melakukan parsing file HTML
            \lstset{style=pythonstyle}
            \begin{lstlisting}[language=python]
#Parsing
soup = BeautifulSoup(html_doc, 'html.parser')
# Formating the parsed html file
strhtm = soup.prettify()

# Mengambil data
print(soup.title)
print(soup.title.string)
print(soup.a.string)
print(soup.b.string)

# Mengambil semua tag tertentu
for x in soup.find_all('a'): print(x.string)
            \end{lstlisting}
        \end{enumerate}
        
    \end{flushleft}

    \newpage
    \begin{flushleft}
        \textbf{Tugas}
        \newline

        \begin{enumerate}
            \item Buatlah program yang dapat mengambil data dari file PDF, Word, JSON, dan HTML
            \item Buatlah program yang dapat mengambil data dari API Twitter
        \end{enumerate}
    \end{flushleft}
\end{document}
